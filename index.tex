% Dokumentanklasse: a4paper, 14pt
% Beschreibung:     Dokumentenformat
% Option:           extraarticle - ?
\documentclass[12pt]{article}

% Packet:           xcolor
% Beschreibung:     Define own color schemas
% Option:
% Restriktion:      Muss vor Paket hyperref geladen werden. Ansonsten werden die Farben der Links nicht genutzt.
\usepackage{xcolor}

% Definiere Farben
\definecolor{blue}{HTML}{5E7FB8}
\definecolor{brown}{HTML}{BA9D5E}
\definecolor{green}{HTML}{79B960}
\definecolor{grey}{HTML}{7C7C7C}
\definecolor{light-grey}{HTML}{D5D5D5}
\definecolor{orange}{HTML}{FF7F00}
\definecolor{red}{HTML}{B80A0A}
\definecolor{violet}{HTML}{B688CB}
\definecolor{white}{HTML}{FFFFFF}
\definecolor{yellow}{HTML}{E2E66C}

% Paket:            setspace
% Beschreibung:     Setz über die optionen den Zeilenabstand
% Optionen:         Möglicher Zeilenabstand
%                   singlespacing = 1,0
%                   onehalfspacing = 1,5
%                   doublespacing = 2,0
% Restriktion:      Muss vor Paket hyperref geladen werden. Ansonsten funktioniert das Paket nicht.
\usepackage[onehalfspacing]{setspace}


% ###########################################################################################################################

% Paket:            appendix
% Beschreibung:     Das Paket dient dazu, ausschließlich das Thema einer Überschrift in das Inhaltsverzeichnis zu überführen
% Option:           appendix - Überführt die Überschriften des Anhangs richtig ins das Inhaltsverzeichnis
\usepackage[titletoc]{appendix}

% Paket:            ansmath
% Beschreibung:     Zum darstellen von mathematischen Formeln
\usepackage{amsmath}

% Paket:            bebel
% Beschreibung:     Stellt das Literatur-, Abbilungs-, Tabellenverzeichnis auf eine beliebige Sprache
% Option:           ngerman
\usepackage[ngerman]{babel}

% Paket:            biblatex
% Beschreibung:     Ermöglicht es, ein Literaturverzeichnis zu führen
% Option:
\usepackage[
  style=authoryear-icomp,    % Zitierstil (https://de.overleaf.com/learn/latex/Biblatex_bibliography_styles)
  isbn=false,                % ISBN nicht anzeigen, gleiches geht mit nahezu allen anderen Feldern
  pagetracker=true,          % ebd. bei wiederholten Angaben (false=ausgeschaltet, page=Seite, spread=Doppelseite, true=automatisch)
  maxbibnames=50,            % maximale Namen, die im Literaturverzeichnis angezeigt werden
  maxcitenames=3,            % maximale Namen, die im Text angezeigt werden, ab 4 wird u.a. nach den ersten Autor angezeigt
  autocite=inline,           % regelt Aussehen für \autocite (inline=\parancite)
  block=space,               % kleiner horizontaler Platz zwischen den Feldern
  backref=true,              % Seiten anzeigen, auf denen die Referenz vorkommt
  backend=biber,
  backrefstyle=three+,       % fasst Seiten zusammen, z.B. S. 2f, 6ff, 7-10
  date=short,                % Datumsformatbackend=biber
]{biblatex}
\setlength{\bibitemsep}{1em}     % Abstand zwischen den Literaturangaben
\setlength{\bibhang}{2em}        % Einzug nach jeweils erster Zeile
\addbibresource{referenzen//bibliothek.bib}

% Paket:            caption
% Beschreibung:     Bietet eine große Auswahl an Gestaltungsmöglichkeiten bezüglich der Beschriftung von Bildern und Tabellen.
\usepackage{caption}

% Paket:            colortbl
% Beschreibung:     Ermöglicht Tabellen, Spalten oder Zellen farbig zu gestalten.
% Befehle:          \columncolor
% Dokumentation:    http://texdoc.net/texmf-dist/doc/latex/colortbl/colortbl.pdf
\usepackage{colortbl}

% Paket:            booktabs
% Beschreibung:     Das Hauptaugenmerk von booktabs liegt dabei auf der Gestaltung der horizontalen Linien innerhalb einer Tabelle.
%                   Was unter anderem daran liegt, dass der Autor des Paketes Simon Fear mehrere Regeln bzw. Empfehlungen für die
%                   Erstellung/Gestaltung einer Tabelle gibt. Aufgrund der ersten zwei Regeln:
%                   1) Verwende nie und nimmer vertikale Linien.
%                   2) Verwende keine doppelten Linien.
% Dokumentation:    https://www.namsu.de/Extra/pakete/Booktabs.html
\usepackage{booktabs}

% Paket:            courier
% Beschreibung:     Lädt das Paket courier für Schriftarten mit fester Breite.
% Befehle:          \ttfamily     Aktiviert Courier füt Tabellen bzw. generelle begin-Blöcke
\usepackage{courier}

% Packet:           csquotes
% Beschreibung:     Muss nach babel, polyglossia, biblatex und inputec geladen werden
% Options:
%   autostyle=true  - Diese Anweisung passt das Zitatdesign so an, dass es zur momentan im Dokument verwendeten Sprache passt
%   autostyle=once  - Ändert das Zitatdesign einmalig, sodass es zur Grundsprache des Dokumentes passt.
%   autostyle=try   - Verhalten sich, sollte multilinguale Unterstützung vorhanden sein, wie true und once, doch werden sie,
%                     für den Fall, dass dies nicht möglich wäre, keine Warnungen erzeugen. (D. h. falls weder babel noch
%                     polyglossia geladen wurden). Die Kurzformautostyleist äquivalent zu autostyle=true
% Dokumentation:    http://mirrors.ibiblio.org/CTAN/info/translations/csquotes/de/csquotes-DE.pdf
\usepackage[autostyle=true]{csquotes}

% Paket:            enumitem
% Beschreibung:     Zeilenabstände bei Aufzählungen definieren
% Option:
\usepackage{enumitem}

% Paket:            eurosym
% Beschreibung:     Bildet das Euro-Zeichen in unterschiedlichen Varianten ab
% Option:
\usepackage{eurosym}

% Paket:            fancyhdr
% Beschreibung:     Ermöglich ein generelles Seitenlayout ein zu stellen mit Kopf und Fußzeile.
\usepackage{fancyhdr}

% Paket:            float
% Beschreibung:     Zum Ausrichten von Tabellen und Spalten bzw. deren Zentrierung
% Option:
% Restriktion:      Muss von Paket hyperref geladen werden. Ansonsten funktioniert das Paket nicht.
\usepackage{float}

% Packet:           framemethod
% Beschreibung:
% Option:
\usepackage[framemethod=tikz]{mdframed}
\mdtheorem[
  linecolor=red,
  frametitlefont=\sffamily\bfseries\color{white},
  frametitlebackgroundcolor=red,
]{warn-popup}{Warnung}[subsection]

\mdtheorem[
  linecolor=orange,
  frametitlefont=\sffamily\bfseries\color{white},
  frametitlebackgroundcolor=orange,
]{info-popup}{Information}[subsection]

\mdtheorem[
  linecolor=green,
  frametitlefont=\sffamily\bfseries\color{white},
  frametitlebackgroundcolor=green,
]{example-popup}{Beispiel}

% Paket:            footmisc
% Beschreibung:     Das footmisc Paket liefert viele Möglichkeiten, wie Fußnoten in Dokumenten dargestellt werden können.
% Option:
% Dokumentation:    http://mirror.utexas.edu/ctan/info/translations/footmisc/de/footmiscDE.pdf
\usepackage[bottom]{footmisc}

% Paket:            geometry
% Beschreibung:     A4 Seitenabstände
% Option:
\usepackage{geometry}
\geometry{
%  a4paper,           % Papierformat (wird auch über die Dokumentenklasse definiert)
  top=2.5cm,          % Abstand Kopfseite   (Zwischen Kopfseite und Inhalt)
  bottom=2cm,         % Abstand Fußseite    (Zwischen Fußseite und Inhalt)
  left=2.5cm,         % Abstand Linkeseite  (Zwischen Linkerseite und Inhalt)
  right=2.5cm,        % Abstand Rechteseite (Zwischen Recherseite und Inhalt)
%  width=              %                    textwidth (+marginsep +marinparwidth)
%  textwidth=15cm,     % Textbreite
%  marginparsep=1cm,   % Randnotiztrennng
%  marginparwidth=10cm,% Randnotizbreite
%  height=             %                    textheight (+headheight +headsep + footskip)
%  textheight=        % Texthöhe
  headheight=1cm,     % Kopfhöhe
  headsep=0.5cm,      % Kopftrennung        (Größe zwischen Kopfzeile und Inhalt)
  footskip=1cm,       % Fußzeilenhöhe
}

% Paket:            graphicx
% Beschreibung:     Einbinden von Bildern
% Option:
\usepackage{graphicx}

% Packet:           Hyperref
% Beschreibung:     Importiert hyperref um Querverweise mittels \hyperref zu erzeugen.
% Dokumentation:    https://www.namsu.de/Extra/pakete/Hyperref.html
\usepackage{hyperref}
\hypersetup{
  colorlinks=false,                 % hyperlinks are coloured
  citecolor=green,                  % color for cite links, only visible if colorlinks=true
  linkcolor=red,                    % color for page links, only visible if colorlinks=true
  urlcolor=orange,                  % color for url links, only visible if colorlinks=true
  citebordercolor=green,            % color for citeborder, only visible if colorlinks=true
  urlbordercolor=orange,            % color for url links, only visible if colorlinks=true
  linkbordercolor=red,              % color for page links, only visible if colorlinks=true
  pdfborderstyle={/S/U/W 1},        % border style will be underline of width 1pt
  pdftitle={Datenbanken WS1819},
  pdfauthor={Markus Pesch},
  pdfsubject={Tutorium},
  pdfkeywords={},
  pdfcreator={pdflatex},
  pdfproducer={LaTeX with hyperref}
}

% Paket:            glossaries
% Beschreibung:     Das Paket glossaries muss nach dem Paket hyperref geladen werden
% Dokumentation:    http://ftp.gwdg.de/pub/ctan/macros/latex/contrib/glossaries/glossaries-user.pdf
% Option/en:
%   acronyms        - This is equivalent to acronym=true and may be used in the document class option list.
%   section         - This is a key=value option. Its value should be the name of a sectional unit (e.g. chapter).
%                     This will make the glossaries appear in the named sectional unit, otherwise each glossary will
%                     appear in a chapter, if chapters exist, otherwise in a section. Unnumbered sectional units will
%                     be used by default.
%   toc             - Add the glossaries to the table of contents.
\usepackage[toc,acronyms]{glossaries}
\makeglossaries
\include{referenzen/glossar}

% Paket:            utf8
% Beschreibung:     Stellt Umlaute richtig dar
% Option:           inputenc - Erlaubt die Darstellung der gleichen Zeichen (Character) wie sie in stdin überliefert werden
\usepackage[utf8]{inputenc}

% Paket:            lipsum
% Beschreibung:
% Option:
\usepackage{lipsum}

% Paket:            makeindex
% Beschreibung:     Ermöglicht das Indexieren von Wörter und den Befehl \printindex um den Index auszugeben
\usepackage{makeidx}
\makeindex

% Packet:           Minted
% Beschreibung:     Dient zum highlining von Quellcode wie beispielsweise Java, Bash oder Python.
% Option/en:
%   autogobble:       Leerzeichen zwischen linken Rand und Sourcecode einrücken bzw. weg schneiden.
%   breaklines:       Automatische Zeilenumbrüche
%   cache:            de- oder aktiviert den cache um Sourcecode zwischen zu speichern und so das PDF schneller zu erzeugen
%   cachedir:         Definiert den Pfad zum cache, an dem minted seine Daten zwischen speichern kann
%   fontfamily:       Die Schriftart die benutzt werden soll. tt, courier und helvetica sind vordefiniert.
%   fontsize:         Die Schriftgröße die benutzt werden soll. Beispielsweise fontsize=\footnotesize
%   linenos:          Zeilennummern
%   keywordcase:      Änderung der Buchstaben. Takes lower, upper, or capitalize.
%   showspaces:       Blendet Leerzeichen ein
\usepackage[cache=true]{minted}

% usemintedstyle
% Gebe 'pygmentize -L styles' im Terminal ein um alle verfügbaren styles anzuzeigen.
\usemintedstyle{tango}

% newminted
% Definiere neue aliase um einmalig ein highlighting pro Sprache zu deklarieren
% \newminted{<makroname>}{optionen} ist verfügbar unter "<makroname>code"
\newminted{awk}{autogobble=true, breaklines=true, linenos=true}
\newminted{json}{autogobble=true, breaklines=true, linenos=true}
\newminted{julia}{autogobble=true, breaklines=true, linenos=true}
\newminted{r}{autogobble=true, breaklines=true, linenos=true}
\newminted{sql}{autogobble=true, breaklines=true, linenos=false, keywordcase=upper}
\newminted{xml}{autogobble=true, breaklines=true, linenos=true}

% newmintedfile
% Definiere neue Makros um automatisch Sourcecode aus Dateien zu highlighten.
% \makroname{Dateipfad}
\newmintedfile[inputawk]{awk}{autogobble=true, breaklines=true, linenos=true}
\newmintedfile[inputjson]{json}{autogobble=true, breaklines=true, linenos=true}
\newmintedfile[inputjulia]{julia}{autogobble=true, breaklines=true, linenos=true}
\newmintedfile[inputr]{r}{autogobble=true, breaklines=true, linenos=true}
\newmintedfile[inputsql]{sql}{autogobble=true, breaklines=true, linenos=true, keywordcase=upper}
\newmintedfile{inputxml}{autogobble=true, breaklines=true, linenos=true}

% newmintinline
% Definiere neues Makro um Sourcecoude einzeiler zu highlighten
% \begin{awkcode} \end{awkcode}
\newmintinline{awk}{autogobble=true, breaklines=true, linenos=true}
\newmintinline{json}{autogobble=true, breaklines=true, linenos=true}
\newmintinline{julia}{autogobble=true, breaklines=true, linenos=true}
\newmintinline{r}{autogobble=true, breaklines=true, linenos=true}
\newmintinline{sql}{autogobble=true, breaklines=true, linenos=true, keywordcase=upper}
\newmintinline{xml}{autogobble=true, breaklines=true, linenos=true}

% Paket:            multirow
% Beschreibung:     Zum kombinieren mehrerer Zellen einer Tabelle
% Option:
\usepackage{multirow}

% Paket:            natbib
% Beschreibung:     Für Zitate
% Option:           round - ?
%\usepackage[round]{natbib}

% Paket:            pdflscape
% Beschreibung:     Ermöglicht Seiten horizontal darzustellen
% Option:           \begin{landscape} \end{landscape}
\usepackage{pdflscape}

% Paket:            showframe
% Beschreibung:     Blendet alle Frames (Textkörper, Fußzeile Kopzeile, Seitenrand) ein
% Option:
% \usepackage{showframe}

% Paket:            subcaption
% Beschreibung:     Bietet eine große Auswahl an Gestaltungsmöglichkeiten bezüglich der Beschriftung von Bildern und Tabellen
%                   die neben oder unter untereinander dargestellt werden sollen.
\usepackage{subcaption}

% format=NAME           - Einstellung des grundsätzlichen Formats von Gleitobjektbeschriftungen
% indention=EINZUG      - Einstellung des Einzugs der Beschriftung ab der zweiten Zeile
% labelformat=NAME      - Einstellung der Zusammensetzung des Bezeichners (z. B. Label und Nummer) der Beschriftung
% labelsep=NAME         - Einstellung des Trenners nach dem Bezeichner
% justification=NAME    - Einstellung er Ausrichtung des Textes Beschriftung
% singlelinecheck=BOOL  - Sonderbehandlung für einzeilige Beschriftungen ein- oder ausschalten
% font=NAME             - Einstellung der Schrift der gesamten Beschriftung
% labelfont=NAME        - Einstellung der Schrift des Bezeichners
% textfont=NAME         - Einstellung der Schrift des Textes der Beschriftung
% margin=BREITE         - Einstellung der Breite des Randes der Beschriftung
% width=BREITE          - Einstellung der Breite der Beschriftung
% parskip=ABSTAND       - Einstellung des Absatzabstandes der Beschriftung
% hangindent=EINZUG     - Einstellung des Einzugs abgesehen von ersten Absatzzeilen
% style=NAME            - Auswahl einer vordefinierten Gesamteinstellung
% aboveskip=ABSTAND     - Einstellung des Abstandes vor einer Beschriftung
% belowskip=ABSTAND     - Einstellung des Abstandes nach einer Beschriftung
% position=WAHL         - Einstellung ob die Beschriftung als Über- oder Unterschrift formatiert werden soll
% tableposition=WAHL    - Einstellung ob nur bei Tabellen die Beschriftung als Über- oder Unterschrift formatiert werden soll
\captionsetup[figure]{position=bottom}
\captionsetup[table]{position=bottom}

% Packet:           tabularx
% Beschreibung:     Werden Tabellen mit diesem Paket erstellt, ist es möglich Zeilenumbrüche in einer Zelle zu erzeugen
\usepackage{tabularx}

% Paket:            textpos
% Beschreibung:     Zum erstellen von Textboxen
% \begin{textblock*}{\textwidth}(0cm,0.5cm)
\usepackage{textpos}

% Paket:            tikz
% Beschreibung:     TikZ and PGF are TeX packages for creating graphics programmatically.
% Dokumentation:
\usepackage{tikz}
\usetikzlibrary{intersections}

% Packet:           varwidth
% Beschreibung:     The package defines a varwidth environment (based on minipage) which is an analogue
%                   of minipage, but whose resulting width is the natural width of its contents.
% Dokumentation:    http://ctan.math.washington.edu/tex-archive/macros/latex/contrib/varwidth/varwidth-doc.pdf
%                   https://www.namsu.de/Extra/befehle/Minipage.html
% Option:
\usepackage{varwidth}

% Paket:            verbatim
% Beschreibung:     Bildet einen Quelltext exakt ab.
% Options:
\usepackage{verbatim}

% Paket:            warapfig
% Beschreibung:     Ermöglich das floaten von Texten neben Bildern
% Options:
% \begin{wrapfigure}{R}{0.30\textwidth}
%   \includegraphics[width=0.30\textwidth]{img/middleware.png}
%   \caption{Middleware}
% \end{wrapfigure}
\usepackage{wrapfig}

% Start des Dokuments
\begin{document}

  % Set Globally Table-Margin
  \def\arraystretch{1.2}

  % Fetch Commit ID and Date
  \immediate\write18{./git-info.sh commit > git-id.tmp}
  \immediate\write18{./git-info.sh date > git-date.tmp}
  \immediate\write18{./git-info.sh url > git-url.tmp}

  % Importiere weitere .tex Dokumente
  \begin{titlepage}
  \begin{center}
    \begin{large}
      Tutorium WS1819
    \end{large}

    \begin{huge}
      \begin{singlespace}
        \textbf{Grundlagen Datenbanken}
      \end{singlespace}
    \end{huge}

    \vspace{0.5cm}

    \begin{figure}[h]
      \centering
      \includegraphics[width=0.85\textwidth]{img//logo.png}
      \label{img:fh-trier-logo}
    \end{figure}

    \vspace{2cm}
    \begin{large}
      \textit{Markus Pesch} \\
      \href{mailto:peschm@hochschule-trier.de}{\textit{peschm@hochschule-trier.de}}
    \end{large}
    \vspace{2cm}

    Version \input{git-id.tmp} vom \input{git-date.tmp}

  \end{center}
\end{titlepage}

  \pagebreak

  % Pagestyle
  % Setze das Seitenlayout auf fancyhdr um Fuß- und Kopfzeilen zu setzen
  \pagestyle{fancy}

  % Löscht alle Kopf- und Fußzeilen des pagestyles fancyhdr
  \fancyhf{}

  % Fuß- und Kopfzeile des Paketes fancyhdr
  % [L] - Linkeseite      [O] - Ungerade Seitenzahlen         [LE,LO] - Linkeseite, Gerade- und Ungerade Seitenanzahlen
  % [C] - Mitte           [E] - Gerade Seitenanzahlen         [CE]    - Seitenmitte, nur gerade Seitenanzahlen
  % [R] - Rechteseite                                         [RO]    - Rechteseite, nur ungerade Seitenanzahlen
  % \fancyhead    Kopfzeile
  % \fancyfoot    Fußzeile
  \fancyhead[L]{\rightmark}
  \fancyhead[R]{\includegraphics[width=4cm]{img/logo.png}}
  \fancyfoot[L]{Grundlagen Datenbanken WS1819 - Tutorium}
  \fancyfoot[C]{}

  % Pixelstärke der Kopf- und Fußzeilenlinie
  \renewcommand{\headrulewidth}{1pt}
  \renewcommand{\footrulewidth}{1pt}

  % Agenda
  \tableofcontents
  \pagebreak

  % Setze die Seitenbeginn zurück
  \setcounter{page}{1}
  \fancyfoot[R]{Seite \thepage}

  % Importiere weitere .tex Dokumente
  % ##########################################################################
% ############################# Übungsblatt 01 #############################
% ##########################################################################
\section{Übungsblatt}
\label{sec:uebung_01}

% ############################### Aufgabe 01 ###############################
\subsection{Aufgabe}
\label{sec:uebung_01.aufgabe_01}
Lade die SQL-Datei \href{https://raw.githubusercontent.com/fh-trier/tgdb_ws1819/master/sql/schema_01.sql}{schema\_01.sql} von \href{https://raw.githubusercontent.com/fh-trier/tgdb_ws1819/master/sql/schema_01.sql}{github.com} auf deinen PC herunter. Starten das SQL-Skript mittels SQL-Plus. Wie sieht zum starten des SQL-Skripts der Befehl in SQL-Plus aus?

\begin{info-popup}
  Nach laden des Schemas können alle Tabellen und Views mit einem Befehl in SQL-Plus ausgegeben werden. \\
  \sqlinline{SELECT * FROM tab;}
\end{info-popup}

\subsubsection*{Lösung}
\label{sec:uebung_01.aufgabe_01.loesung}
\inputsql{sql/uebung_01/aufgabe_01.sql}

% ############################### Aufgabe 02 ###############################
\subsection{Aufgabe}
\label{sec:uebung_01.aufgabe_02}
Gebe dir alle Datensätze aus der Tablle \texttt{EMPLOYEES} aus.

\begin{info-popup}
Du kannst die Ausgabe in SQL-Plus anpassen. Nutze dazu den SQL-Plus Befehl
\texttt{COLUMN <spalte> FORMAT <format>}

Beispielsweise: \texttt{COLUMN EMPLOYEE\_FIRSTNAME FORMAT a20}.

Eine vordefinierte Formatierungseinstellung ist auf \href{https://raw.githubusercontent.com/fh-trier/tgdb_ws1819/master/sql/sqlplus-settings.sql}{github.com} hinterlegt.

\end{info-popup}

\subsubsection*{Lösung}
\label{sec:uebung_01.aufgabe_02.loesung}
\inputsql{sql/uebung_01/aufgabe_02.sql}

% ############################### Aufgabe 03 ###############################
\subsection{Aufgabe}
\label{sec:uebung_01.aufgabe_03}
Modifiziere Aufgabe \ref{sec:uebung_01.aufgabe_02} so, dass ein ähnliches Ergebnis zurück geliefert wird wie in Abbildung \ref{tbl:uebung_01.aufgabe_03}.

\begin{table}[H]
  \ttfamily
  \begin{tabularx}{\textwidth}{X|X|X|X}
    \textbf{FIRSTNAME} & \textbf{LASTNAME} & \textbf{BIRTHDAY} & \textbf{HIREDATE} \\
    \hline\hline
    Maximilian & Arbeitsscheu & 1998-06-21 & 2007-04-18 \\
    Henry & Großkreutz & 1990-09-01 & 2009-02-10 \\
    Leni & Mayer & 1996-10-15 & 2009-02-10 \\
    $[$\dots$]$ & $[$\dots$]$ & $[$\dots$]$ & $[$\dots$]$ \\
  \end{tabularx}
  \caption{Beispielhaftes Abfrageergebnis zu Aufgabe \ref{sec:uebung_01.aufgabe_03}}
  \label{tbl:uebung_01.aufgabe_03}
\end{table}

\subsubsection*{Lösung}
\label{sec:uebung_01.aufgabe_03.loesung}
\inputsql{sql/uebung_01/aufgabe_03.sql}

% ############################### Aufgabe 04 ###############################
\subsection{Aufgabe}
\label{sec:uebung_01.aufgabe_04}
Modifiziere Aufgabe \ref{sec:uebung_01.aufgabe_03} so, dass nur Mitarbeiter angezeigt werden, deren Nachname mit einem \textbf{A} beginnt.

\subsubsection*{Lösung}
\label{sec:uebung_01.aufgabe_04.loesung}
\inputsql{sql/uebung_01/aufgabe_04.sql}

% ############################### Aufgabe 05 ###############################
\subsection{Aufgabe}
\label{sec:uebung_01.aufgabe_05}
Gebe alle Mitarbeiter aus, die an vierter Stelle ihres Nachnamens ein \textbf{i} oder ein \textbf{ß} besitzen.

\subsubsection*{Lösung}
\label{sec:uebung_01.aufgabe_05.loesung}
\inputsql{sql/uebung_01/aufgabe_05.sql}

% ############################### Aufgabe 06 ###############################
\subsection{Aufgabe}
\label{sec:uebung_01.aufgabe_06}
Gebe alle Mitarbeiter aus, die in den 90er Jahren geboren sind.

\begin{info-popup}
  Nutze die Funktion \sqlinline{TO_CHAR(datum, 'format')} um ein Datum in einen CHAR-Value zu formatieren - \href{https://docs.oracle.com/cd/B19306_01/server.102/b14200/functions180.htm}{Dokumentation}.

\end{info-popup}

\subsubsection*{Lösung}
\label{sec:uebung_01.aufgabe_06.loesung}
\inputsql{sql/uebung_01/aufgabe_06.sql}

% ############################### Aufgabe 07 ###############################
\subsection{Aufgabe}
\label{sec:uebung_01.aufgabe_07}
Gebe alle Kontakte aus, die der Firma \textbf{Thyssen} angehören und suche nach Personen, deren Telefonnummer auf \textbf{0211} endet.

\subsubsection*{Lösung}
\label{sec:uebung_01.aufgabe_07.loesung}
\inputsql{sql/uebung_01/aufgabe_07.sql}

% ############################### Aufgabe 08 ###############################
\subsection{Aufgabe}
\label{sec:uebung_01.aufgabe_08}
Welchem Unternehmen gehört \textit{Lena Hörnchen} an?

\subsubsection*{Lösung}
\label{sec:uebung_01.aufgabe_08.loesung}
\inputsql{sql/uebung_01/aufgabe_08.sql}

% ############################### Aufgabe 09 ###############################
\subsection{Aufgabe}
\label{sec:uebung_01.aufgabe_09}
Lasse dir mittels \sqlinline{WHERE ROWNUM <= 10} in der \texttt{WHERE}-Bedingung eines \texttt{SELECT}-Statements die ersten 10 Datensätze aus der Tabelle \texttt{CONTACTS} anzeigen. Was fällt dir auf?

\subsubsection*{Lösung}
\label{sec:uebung_01.aufgabe_09.loesung}
\inputsql{sql/uebung_01/aufgabe_09.sql}

% ############################### Aufgabe 10 ###############################
\subsection{Aufgabe}
\label{sec:uebung_01.aufgabe_10}
Lasse dir alle Kontakte ausgeben die keine E-Mail Adresse haben.

\subsubsection*{Lösung}
\label{sec:uebung_01.aufgabe_10.loesung}
\inputsql{sql/uebung_01/aufgabe_10.sql}

% ############################### Aufgabe 11 ###############################
\subsection{Aufgabe}
\label{sec:uebung_01.aufgabe_11}
Gebe eine Liste aus, in welchen Regalen das Material \textbf{Edelstahl} vom Typ \textbf{1.4301, Geschliffen} vorhanden ist. Eine Beispielausgabe findest du unter der Tabelle \ref{tbl:uebung_01.aufgabe_11}

\begin{table}[H]
  \ttfamily
  \begin{tabularx}{\textwidth}{X|X}
    \textbf{SHELF\_NAME} & \textbf{SHELF\_NUMBER} \\
    \hline\hline
    A & 30 \\
    A & 50 \\
    B & 20 \\
    $[$\dots$]$ & $[$\dots$]$ \\
  \end{tabularx}
  \caption{Examplarisches Abfrageergebnis für Aufgabe \ref{sec:uebung_01.aufgabe_11}}
  \label{tbl:uebung_01.aufgabe_11}
\end{table}

\subsubsection*{Lösung}
\label{sec:uebung_01.aufgabe_11.loesung}
\inputsql{sql/uebung_01/aufgabe_11.sql}


% ############################### Aufgabe 12 ###############################
\subsection{Aufgabe}
\label{sec:uebung_01.aufgabe_12}
Versuche die Spalten aus Aufgabe \ref{sec:uebung_01.aufgabe_11} zu einer Spalte zu konkatenieren - \href{https://docs.oracle.com/cd/B19306_01/server.102/b14200/functions026.htm}{Dokumentation}.

\subsubsection*{Lösung}
\label{sec:uebung_01.aufgabe_12.loesung}
\inputsql{sql/uebung_01/aufgabe_12.sql}

% ############################### Aufgabe 13 ###############################
\subsection{Aufgabe}
\label{sec:uebung_01.aufgabe_13}
Versuche die Aufgabe \ref{sec:uebung_01.aufgabe_11} mit einer \sqlinline{WHERE}-Bedingungen zu lösen.

\subsubsection*{Lösung}
\label{sec:uebung_01.aufgabe_13.loesung}
\inputsql{sql/uebung_01/aufgabe_13.sql}

% ############################### Aufgabe 14 ###############################
\subsection{Aufgabe}
\label{sec:uebung_01.aufgabe_14}
Welche Firmen haben ihren Sitz im Ausland?

\subsubsection*{Lösung}
\label{sec:uebung_01.aufgabe_14.loesung}
\inputsql{sql/uebung_01/aufgabe_14.sql}

% ############################### Aufgabe 15 ###############################
\subsection{Aufgabe}
\label{sec:uebung_01.aufgabe_15}
Welche Mitarbeiter haben noch keinen Eintrag in der Tabelle \texttt{STORAGE}? Löse die Aufgabe per \texttt{JOIN}-Statement.

\subsubsection*{Lösung}
\label{sec:uebung_01.aufgabe_15.loesung}
\inputsql{sql/uebung_01/aufgabe_15.sql}

  % ##########################################################################
% ############################# Übungsblatt 02 #############################
% ##########################################################################
\section{Übungsblatt}
\label{sec:uebung_02}

% ############################### Aufgabe 01 ###############################
\subsection{Aufgabe}
\label{sec:uebung_02.aufgabe_01}
Verschaffe dir einen Überblick über das \hyperref[app:er-diagramm]{Datenbankmodell}. Dazu gehören die Beziehungen der Tabellen untereinander.


% ############################### Aufgabe 02 ###############################
\subsection{Aufgabe}
\label{sec:uebung_02.aufgabe_02}
Füge in die Tabelle \texttt{EMPLOYEES} drei neue Mitarbeiter ein. Nutze um einen Primary-Key zu erzeugen die Funktion \nameref{app:funktionen.new_uuid}.

Die Funktion lässt sich in \texttt{INSERT}-Statements per \texttt{NEW\_UUID()} aufrufen.

\subsubsection*{Lösung}
\label{sec:uebung_02.aufgabe_02.loesung}
\inputsql{sql/uebung_02/aufgabe_02.sql}

% ############################### Aufgabe 03 ###############################
\subsection{Aufgabe}
\label{sec:uebung_02.aufgabe_03}
Füge ein neues Regal mit dem Namen X50 und X100 in die Tabelle \texttt{SHELFS} ein. Nutze wie in Aufgabe \ref{sec:uebung_02.aufgabe_02} die \nameref{app:funktionen.new_uuid} zum erzeugen eines Primary Keys.

\subsubsection*{Lösung}
\label{sec:uebung_02.aufgabe_03.loesung}
\inputsql{sql/uebung_02/aufgabe_03.sql}

% ############################### Aufgabe 04 ###############################
\subsection{Aufgabe}
\label{sec:uebung_02.aufgabe_04}
Füge das Metall Titan als Typ Geschliffen und Träne als auch Gold hinzu.

\subsubsection*{Lösung}
\label{sec:uebung_02.aufgabe_04.loesung}
\inputsql{sql/uebung_02/aufgabe_04.sql}

% ############################### Aufgabe 05 ###############################
\subsection{Aufgabe}
\label{sec:uebung_02.aufgabe_05}
Füge den Firmenvertreter Michael Jakobs von der Firma Willems Metallhandel GmbH hinzu.

\subsubsection*{Lösung}
\label{sec:uebung_02.aufgabe_05.loesung}
\inputsql{sql/uebung_02/aufgabe_05.sql}

% ############################### Aufgabe 06 ###############################
\subsection{Aufgabe}
\label{sec:uebung_02.aufgabe_06}
Einer deiner Kollegen, die in Aufgabe \ref{sec:uebung_02.aufgabe_02} hinzugefügt wurden kauften bei Michael Jakobs am 14. Juli 2018 zwei Titanplatten vom Typ Geschliffen in der Größe 5000x4000x3mm. Die zwei Platten befinden sich im Lager X100. Eine weitere Titanplatte vom Typ Träne in der Größe 2500x1000x1mm wurde über die Firma  Willems Metallhandel GmbH am 17. August 2018 bezogen und in das Lager X50 eingelagert.

\begin{info-popup}
  Die Spalte \texttt{SURFACE\_AREA} braucht nicht befüllt zu werden. Es handelt sich bei der Spalte um eine \href{https://oracle-base.com/articles/11g/virtual-columns-11gr1}{Virtual-Column}, die die Fläche der Platte oder des Bleches in $mm^2$ berechnet.
\end{info-popup}

\subsubsection*{Lösung}
\label{sec:uebung_02.aufgabe_06.loesung}
\inputsql{sql/uebung_02/aufgabe_06.sql}

% ############################### Aufgabe 07 ###############################
\subsection{Aufgabe}
\label{sec:uebung_02.aufgabe_07}
Es ist ein Fehler unterlaufen. Michael Jakos konnte niemals vor dem 1. August im Namen seiner Firma Waren verkaufen, da dieser erst zum 1. August eingestellt wurde.

Ändere den Ansprechpartner alle Lageraufträge die vor dem 1. August durch Michael Jakobs getätigt wurden auf Sabrina Böckel ab. Sabrina Böckels Handynummer is die +49 0554 788112784, die E-Mail ist sabrina.boeckel@willems-gmbh.de und die Festnetznummer lautet +49 7854 22145 985.

\subsubsection*{Lösung}
\label{sec:uebung_02.aufgabe_07.loesung}
\inputsql{sql/uebung_02/aufgabe_07.sql}

% ############################### Aufgabe 08 ###############################
\subsection{Aufgabe}
\label{sec:uebung_02.aufgabe_08}
Lösche alle Lagereinträge, bei denen es sich um Aluminium Geschliffen im Regal A80 handelt, die von Lena Hörnchen hinzugefügt wurden.

\subsubsection*{Lösung}
\label{sec:uebung_02.aufgabe_08.loesung}
\inputsql{sql/uebung_02/aufgabe_08.sql}

  % ##########################################################################
% ############################# Übungsblatt 03 #############################
% ##########################################################################
\section{Übungsblatt}
\label{sec:uebung_03}

% ############################### Aufgabe 01 ###############################
\subsection{Aufgabe}
\label{sec:uebung_03.aufgabe_01}
Gebe alle Rollen der aktuellen Session aus.

\subsubsection*{Lösung}
\label{sec:uebung_03.aufgabe_01.loesung}
\inputsql{sql/uebung_03/aufgabe_01.sql}

% ############################### Aufgabe 02 ###############################
\subsection{Aufgabe}
\label{sec:uebung_03.aufgabe_02}
Ermittle warum du \texttt{INSERT}-Rechte auf die Tabelle \texttt{PESCHM.CONTACTS} und \texttt{UPDATE}-Rechte auf die Tabelle \texttt{PESCHM.STORAGE} besitzt.
Dabei sollen folgende Fragen beantwortet werden:

\begin{itemize}[itemsep=0pt]
  \item Wurden die Tabellen-Rechte direkt an dich oder an \texttt{PUBLIC} vergeben?
  \item Welche Rollen besitzt du direkt? Unterscheidet sich zu Aufgabe \ref{sec:uebung_03.aufgabe_01}
  \item Welche Rollen sind anderen Rollen zugeordnet?
  \item Haben Rollen Rechte an \texttt{PESCHM.CONTACTS} oder \texttt{PESCHM.STORAGE}?
\end{itemize}

\subsubsection*{Lösung}
\label{sec:uebung_03.aufgabe_02.loesung}
\inputsql{sql/uebung_03/aufgabe_02.sql}

% ############################### Aufgabe 03 ###############################
\subsection{Aufgabe}
\label{sec:uebung_03.aufgabe_03}
Füge einen neuen Kontakt in die Tabelle \texttt{PESCHM.CONTACTS} ein.

\subsubsection*{Lösung}
\label{sec:uebung_03.aufgabe_03.loesung}
\inputsql{sql/uebung_03/aufgabe_03.sql}

% ############################### Aufgabe 04 ###############################
\subsection{Aufgabe}
\label{sec:uebung_03.aufgabe_04}
Verkürze den Namen der Ressource \texttt{PESCHM.CONTACTS} durch ein \texttt{SYNONYM}.

\subsubsection*{Lösung}
\label{sec:uebung_03.aufgabe_04.loesung}
\inputsql{sql/uebung_03/aufgabe_04.sql}

% ############################### Aufgabe 05 ###############################
\subsection{Aufgabe}
\label{sec:uebung_03.aufgabe_05}
Gebe alle Datensätze mit den Spalten \texttt{FIRSTNAME}, \texttt{LASTNAME}, \texttt{PHONE}, \texttt{MOBILE} und \texttt{MAIL} der Tabelle \texttt{CONTACTS} und der Tabelle \texttt{PESCHM.CONTACTS} unter Angabe deiner verkürzten Schreibweise aus Aufgabe \ref{sec:uebung_03.aufgabe_04} aus. Vermeide bei der Ausgabe doppelte Datensätze.

\subsubsection*{Lösung}
\label{sec:uebung_03.aufgabe_05.loesung}
\inputsql{sql/uebung_03/aufgabe_05.sql}

% ############################### Aufgabe 06 ###############################
\subsection{Aufgabe}
\label{sec:uebung_03.aufgabe_06}
Speichere die Abfrage aus Aufgabe \ref{sec:uebung_03.aufgabe_05} als \texttt{VIEW}.

\subsubsection*{Lösung}
\label{sec:uebung_03.aufgabe_06.loesung}
\inputsql{sql/uebung_03/aufgabe_06.sql}

% ############################### Aufgabe 07 ###############################
\subsection{Aufgabe}
\label{sec:uebung_03.aufgabe_07}
Räume dem Benutzer \texttt{PESCHM} und der Rolle \texttt{FH\_TRIER} das Recht ein, Datensätze aus dem Lager zu abzurufen. Dabei soll der Benutzer \texttt{PESCHM} zusätzlich die Spalte \texttt{CONTACT\_ID} aktualisieren dürfen.

\subsubsection*{Lösung}
\label{sec:uebung_03.aufgabe_07.loesung}
\inputsql{sql/uebung_03/aufgabe_07.sql}

% ############################### Aufgabe 07 ###############################
\subsection{Aufgabe}
\label{sec:uebung_03.aufgabe_08}
Entziehe Benutzer \texttt{PESCHM} und der Rolle \texttt{FH\_TRIER} die unter Aufgabe \ref{sec:uebung_03.aufgabe_07} eingeräumten Rechte.

\subsubsection*{Lösung}
\label{sec:uebung_03.aufgabe_08.loesung}
\inputsql{sql/uebung_03/aufgabe_08.sql}

  % ##########################################################################
% ############################# Übungsblatt 04 #############################
% ##########################################################################
\section{Übungsblatt}
\label{sec:uebung_04}

% ############################### Aufgabe 01 ###############################
\subsection{Aufgabe}
\label{sec:uebung_04.aufgabe_01}
Lassen Sie sich den Inhalt der View \texttt{JOINED\_STORAGE} ausgeben. Erweitern Sie die Abfrage um die Spalten \texttt{EMPLOYEE\_FIRSTNAME} und \texttt{EMPLOYEE\_LASTNAME}.

\subsubsection*{Lösung}
\label{sec:uebung_04.aufgabe_01.loesung}
\inputsql{sql/uebung_04/aufgabe_01.sql}

% ############################### Aufgabe 02 ###############################
\subsection{Aufgabe}
\label{sec:uebung_04.aufgabe_02}
Geben sie die Anzahl aus, wie viele gruppierte Einträge jeder Mitarbeiter im Lager hinterlegt hat.

\subsubsection*{Lösung}
\label{sec:uebung_04.aufgabe_02.loesung}
\inputsql{sql/uebung_04/aufgabe_02.sql}

% ############################### Aufgabe 03 ###############################
\subsection{Aufgabe}
\label{sec:uebung_04.aufgabe_03}
Ermitteln Sie den Durchschnitt über alle gruppierten Einträge pro Mitarbeiter im Lager.

\subsubsection*{Lösung}
\label{sec:uebung_04.aufgabe_03.loesung}
\inputsql{sql/uebung_04/aufgabe_03.sql}

% ############################### Aufgabe 04 ###############################
\subsection{Aufgabe}
\label{sec:uebung_04.aufgabe_04}
Sie haben \texttt{SELECT}-Berechtigungen auf die Tabelle \texttt{PESCHM.METALS}. Lassen Sie sich den Inhalt der Tabelle oder die Beschreibung der Tabelle ausgeben. Zu erkennen ist, das die Tabelle eine neue Spalte \texttt{METAL\_PRICE} enthält. Dieser Preis wird pro $1000mm\times1000mm\times1mm$ angesetzt. Berechnen Sie den Wert des Lagers.

\begin{info-popup}
  Anstatt die Tabelle \texttt{PESCHM.METALS} zu nutzen, können Sie auch einen Patch des Schemas nachladen. Laden Sie sich dazu die Datei \href{https://raw.githubusercontent.com/fh-trier/tgdb_ws1819/master/sql/patch_01.sql}{patch\_01.sql} herunter und starten Sie diesen mittels SQL-Plus.
\end{info-popup}

\subsubsection*{Lösung}
\label{sec:uebung_04.aufgabe_04.loesung}
\inputsql{sql/uebung_04/aufgabe_04.sql}

% ############################### Aufgabe 05 ###############################
\subsection{Aufgabe}
\label{sec:uebung_04.aufgabe_05}
Ermitteln Sie anhand der zusätzlichen Spalte \texttt{METAL\_PRICE} aus Aufgabe \ref{sec:uebung_04.aufgabe_04} den Lagerwert pro Material. Ergänzen Sie die View aus Aufgabe \ref{sec:uebung_04.aufgabe_01}.

\subsubsection*{Lösung}
\label{sec:uebung_04.aufgabe_05.loesung}
\inputsql{sql/uebung_04/aufgabe_05.sql}

% ############################### Aufgabe 06 ###############################
\subsection{Aufgabe}
\label{sec:uebung_04.aufgabe_06}
Geben Sie alle Datensätze aus, deren Wert größer ist als der Durchschnittspreis des Lagers.

\subsubsection*{Lösung}
\label{sec:uebung_04.aufgabe_06.loesung}
\inputsql{sql/uebung_04/aufgabe_06.sql}

% ############################### Aufgabe 07 ###############################
\subsection{Aufgabe}
\label{sec:uebung_04.aufgabe_07}
Welche/r Mitarbeiter haben/hat den größten Warenwert im Lager verbucht?

\subsubsection*{Lösung}
\label{sec:uebung_04.aufgabe_07.loesung}
\inputsql{sql/uebung_04/aufgabe_07.sql}

% ############################### Aufgabe 08 ###############################
\subsection{Aufgabe}
\label{sec:uebung_04.aufgabe_08}
Welcher Firma hat laut dem Lagerbestand den geringsten Umsatz erzielt. Lassen Sie sich alle Ansprechpartner dieser Firma ausgeben.

\subsubsection*{Lösung}
\label{sec:uebung_04.aufgabe_08.loesung}
\inputsql{sql/uebung_04/aufgabe_08.sql}

% ############################### Aufgabe 09 ###############################
\subsection{Aufgabe}
\label{sec:uebung_04.aufgabe_09}
Ermitteln Sie den Lagerwert pro Monat/Jahr und ordnen Sie die Ausgabe entsprechend der Gruppierung.

\begin{info-popup}
  Wenn alle Datumsangaben identisch sind bzw. im gleichen Monat/Jahr liegen, dann darf nur eine Zeile zurückgegeben werden. Der Wert des Lagers ist identisch mit dem Wert aus Aufgabe \ref{sec:uebung_04.aufgabe_04}.
\end{info-popup}

\subsubsection*{Lösung}
\label{sec:uebung_04.aufgabe_09.loesung}
\inputsql{sql/uebung_04/aufgabe_09.sql}

% ############################### Aufgabe 10 ###############################
\subsection{Knobelaufgabe}
\label{sec:uebung_04.aufgabe_10}
Geben Sie mit einem SQL Befehl alle Klausuren aus, zu denen sich Personen angemeldet haben, aber bei ALLEN angemeldeten Personen fehlt die Note.

\textit{Diese Frage bezieht sich auf das Schema aus der Vorlesung.}

\subsubsection*{Lösung}
\label{sec:uebung_04.aufgabe_10.loesung}
\inputsql{sql/uebung_04/aufgabe_10.sql}
  % ##########################################################################
% ############################# Übungsblatt 05 #############################
% ##########################################################################
\section{Übungsblatt}
\label{sec:uebung_05}

% ############################### Aufgabe 01 ###############################
\subsection{Aufgabe}
\label{sec:uebung_05.aufgabe_01}
Sichere alle Spalten mit einem Primärschlüssel so ab, dass nur eine UUID verwendet werden kann.
Hier nochmal exemplarisch eine UUID: \texttt{ff3d445a-550f-4080-aa61-cf7f29d6852c}

\subsubsection*{Lösung}
\label{sec:uebung_05.aufgabe_01.loesung}
\inputsql{sql/uebung_05/aufgabe_01.sql}

% ############################### Aufgabe 02 ###############################
\subsection{Aufgabe}
\label{sec:uebung_05.aufgabe_02}
Sichern Sie die Tabelle \texttt{METALS} ab, dabei erhält jede Bedingung einen eigenen Constraint.
\begin{itemize}
  \item Vor negativen Preisen
  \item Beschränkung der Metalltypen auf NULL, "Träne" und "Geschliffen"
  \item Beschränkung der Metalleigenschaften auf NULL, "1.4301", "1.6582", "1.4452", "S235" und "S355"
\end{itemize}

\subsubsection*{Lösung}
\label{sec:uebung_05.aufgabe_02.loesung}
\inputsql{sql/uebung_05/aufgabe_02.sql}

% ############################### Aufgabe 03 ###############################
\subsection{Aufgabe}
\label{sec:uebung_05.aufgabe_03}
Sichern Sie den Namen von Regalen so ab, dass nur Großbuchstaben und keine negativen Zahlen vorkommen können. Jede Bedingung wird als eigener Constraint definiert.

\subsubsection*{Lösung}
\label{sec:uebung_05.aufgabe_03.loesung}
\inputsql{sql/uebung_05/aufgabe_03.sql}

% ############################### Aufgabe 04 ###############################
\subsection{Aufgabe}
\label{sec:uebung_05.aufgabe_04}
Vermeiden Sie redundante E-Mail Adressen von Kontakten.

\subsubsection*{Lösung}
\label{sec:uebung_05.aufgabe_04.loesung}
\inputsql{sql/uebung_05/aufgabe_04.sql}

% ############################### Aufgabe 05 ###############################
\subsection{Aufgabe}
\label{sec:uebung_05.aufgabe_05}
Erzeugen Sie einen zusätzlichen Index für die E-Mail Adressen von Kontakten.

\subsubsection*{Lösung}
\label{sec:uebung_05.aufgabe_05.loesung}
\inputsql{sql/uebung_05/aufgabe_05.sql}

% ############################### Aufgabe 06 ###############################
\subsection{Aufgabe}
\label{sec:uebung_05.aufgabe_06}
Erstellen Sie zwischen Ansprechpartnern (\texttt{CONTACTS}) und Lieferanten \texttt{SUPPLIERS} eine Beziehung. Entscheiden Sie über das Löschverhalten dieser Beziehung selbst.

\subsubsection*{Lösung}
\label{sec:uebung_05.aufgabe_06.loesung}
\inputsql{sql/uebung_05/aufgabe_06.sql}


  % Aufzählung aller Bilder
  % \listoffigures
  % \newpage

  % Aufzählung aller Tabellen
  % \listoftables
  % \newpage

  % Glossary
  % \printglossaries
  % \newpage

  % Literatur
  % \printbibliography[heading=bibintoc]

  \appendix
\section{Funktionen}
\label{app:funktionen}

\subsection{NEW\_UUID}
\label{app:funktionen.new_uuid}

\inputsql{sql/funktionen/new_uuid.sql}
\end{document}
