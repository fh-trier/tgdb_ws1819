% ##########################################################################
% ############################# Übungsblatt 01 #############################
% ##########################################################################
\section{Übungsblatt}
\label{sec:uebung_01}

% ############################### Aufgabe 01 ###############################
\subsection{Aufgabe}
\label{sec:uebung_01.aufgabe_01}
Lade die SQL-Datei \href{https://raw.githubusercontent.com/fh-trier/tgdb_ws1819/master/sql/schema_01.sql}{schema\_01.sql} von \href{https://raw.githubusercontent.com/fh-trier/tgdb_ws1819/master/sql/schema_01.sql}{github.com} auf deinen PC herunter. Starten das SQL-Skript mittels SQL-Plus. Wie sieht zum starten des SQL-Skripts der Befehl in SQL-Plus aus?

\subsubsection*{Lösung}
\label{sec:uebung_01.aufgabe_01.loesung}
\inputsql{sql/uebung_01/aufgabe_01.sql}

% ############################### Aufgabe 02 ###############################
\subsection{Aufgabe}
\label{sec:uebung_01.aufgabe_02}
Gebe dir alle Datensätze aus der Tablle \texttt{EMPLOYEES} aus.

\subsubsection*{Lösung}
\label{sec:uebung_01.aufgabe_02.loesung}
\inputsql{sql/uebung_01/aufgabe_02.sql}

% ############################### Aufgabe 03 ###############################
\subsection{Aufgabe}
\label{sec:uebung_01.aufgabe_03}
Modifiziere Aufgabe \ref{sec:uebung_01.aufgabe_02} so, dass ein ähnliches Ergebnis wie in Abbildung \ref{tbl:uebung_01.aufgabe_03} abgebildet ist, zurück geliefert wird.

\begin{table}[H]
  \begin{tabularx}{\textwidth}{X|X|X|X}
    \textbf{FIRSTNAME} & \textbf{LASTNAME} & \textbf{BIRTHDAY} & \textbf{HIREDATE} \\
    \hline\hline
    Maximilian & Arbeitsscheu & 1998-06-21 & 2007-04-18 \\
    Henry & Großkreutz & 1990-09-01 & 2009-02-10 \\
    Leni & Mayer & 1996-10-15 & 2009-02-10 \\
    $[$\dots$]$ & $[$\dots$]$ & $[$\dots$]$ & $[$\dots$]$ \\
  \end{tabularx}
  \caption{Abfrageergebnis Aufgabe \ref{sec:uebung_01.aufgabe_03}}
  \label{tbl:uebung_01.aufgabe_03}
\end{table}

\subsubsection*{Lösung}
\label{sec:uebung_01.aufgabe_03.loesung}
\inputsql{sql/uebung_01/aufgabe_03.sql}

% ############################### Aufgabe 04 ###############################
\subsection{Aufgabe}
\label{sec:uebung_01.aufgabe_04}
Modifiziere Aufgabe \ref{sec:uebung_01.aufgabe_03} so, dass zusätzlich zu jedem Mitarbeiter die Stadt, die Postleitzahl, das Bundesland und das Land mit angezeigt wird.

\subsubsection*{Lösung}
\label{sec:uebung_01.aufgabe_04.loesung}
\inputsql{sql/uebung_01/aufgabe_04.sql}

% ############################### Aufgabe 05 ###############################
\subsection{Aufgabe}
\label{sec:uebung_01.aufgabe_05}
Gebe alle Länder und Ihre Bundesländer aus. Das Ergebnis der Abfrage sollte ähnlich dem aus Tabelle \ref{tbl:uebung_01.aufgabe_05} entsprechen.

\begin{table}[H]
  \begin{tabularx}{\textwidth}{X|X}
    \textbf{Land} & \textbf{Bundesland} \\
    \hline\hline
    Deutschland & Rheinland-Pfalz \\
    Deutschland & Nordrhein-Westfalen \\
    Deutschland & Saarland \\
    $[$\dots$]$ & $[$\dots$]$ \\
  \end{tabularx}
  \caption{Abfrageergebnis Aufgabe \ref{sec:uebung_01.aufgabe_05}}
  \label{tbl:uebung_01.aufgabe_05}
\end{table}

\subsubsection*{Lösung}
\label{sec:uebung_01.aufgabe_05.loesung}
\inputsql{sql/uebung_01/aufgabe_05.sql}

% ############################### Aufgabe 06 ###############################
\subsection{Aufgabe}
\label{sec:uebung_01.aufgabe_06}
Gebe alle Vertreter aus, die bereits Materialien an das Unternehmen verkauft haben.

\subsubsection*{Lösung}
\label{sec:uebung_01.aufgabe_06.loesung}
\inputsql{sql/uebung_01/aufgabe_06.sql}

% ############################### Aufgabe 07 ###############################
\subsection{Aufgabe}
\label{sec:uebung_01.aufgabe_07}
Modifiziere Aufgabe \ref{sec:uebung_01.aufgabe_06} so, dass alle Materialien ausgegeben werden, die der Mitarbeiter \textit{Maximilian Arbeitsscheu} von \textit{Lena Hörnchen} bezogen hat.

\subsubsection*{Lösung}
\label{sec:uebung_01.aufgabe_07.loesung}
\inputsql{sql/uebung_01/aufgabe_07.sql}

% ############################### Aufgabe 08 ###############################
\subsection{Aufgabe}
\label{sec:uebung_01.aufgabe_08}
Welchem Unternehmen gehört \textit{Lena Hörnchen} an?

\subsubsection*{Lösung}
\label{sec:uebung_01.aufgabe_08.loesung}
\inputsql{sql/uebung_01/aufgabe_08.sql}

% ############################### Aufgabe 09 ###############################
\subsection{Aufgabe}
\label{sec:uebung_01.aufgabe_09}
Welchem Personen arbeiten noch für das Unternehmen von \textit{Lena Hörnchen}?

\subsubsection*{Lösung}
\label{sec:uebung_01.aufgabe_09.loesung}
\inputsql{sql/uebung_01/aufgabe_09.sql}

% ############################### Aufgabe 10 ###############################
\subsection{Aufgabe}
\label{sec:uebung_01.aufgabe_10}
Versuche die Aufgabe \ref{sec:uebung_01.aufgabe_09} mittels \textit{Sub-Query} zu lösen.

\subsubsection*{Lösung}
\label{sec:uebung_01.aufgabe_10.loesung}
\inputsql{sql/uebung_01/aufgabe_10.sql}

% ############################### Aufgabe 11 ###############################
\subsection{Aufgabe}
\label{sec:uebung_01.aufgabe_11}
Füge dich als neuen Mitarbeiter in die Tabelle \texttt{EMPLOYEES} ein. Nutze die Funktion \texttt{NEW\_UUID()} (siehe Anhang \ref{app:funktionen.new_uuid}) um eine neue UUID zu generieren.

\subsubsection*{Lösung}
\label{sec:uebung_01.aufgabe_11.loesung}
\inputsql{sql/uebung_01/aufgabe_11.sql}

% ############################### Aufgabe 12 ###############################
\subsection{Aufgabe}
\label{sec:uebung_01.aufgabe_12}
Der Geschäftsführer möchte, dass bei dem Lieferanten \textit{Edelmetalle Benelux GBR} Stahlbleche in der Größe von \texttt{5000mm x 2500mm x 5mm - LxBxT} beschafft werden. Bei den Stahlblechen handelt es sich um \textit{Schwarzstahl} vom Typ \textit{S255}. Wähle ein freies Lager um die Bleche zu Lagern.

Wie sieht für den oben beschrieben Vorgang der SQL-Befehl aus?

\subsubsection*{Lösung}
\label{sec:uebung_01.aufgabe_12.loesung}
\inputsql{sql/uebung_01/aufgabe_12.sql}
