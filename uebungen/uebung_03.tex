% ##########################################################################
% ############################# Übungsblatt 03 #############################
% ##########################################################################
\section{Übungsblatt}
\label{sec:uebung_03}

% ############################### Aufgabe 01 ###############################
\subsection{Aufgabe}
\label{sec:uebung_03.aufgabe_01}
Gebe alle Rollen der aktuellen Session aus.

\subsubsection*{Lösung}
\label{sec:uebung_03.aufgabe_01.loesung}
\inputsql{sql/uebung_03/aufgabe_01.sql}

% ############################### Aufgabe 02 ###############################
\subsection{Aufgabe}
\label{sec:uebung_03.aufgabe_02}
Ermittle warum du \texttt{INSERT}-Rechte auf die Tabelle \texttt{PESCHM.CONTACTS} und \texttt{UPDATE}-Rechte auf die Tabelle \texttt{PESCHM.STORAGE} besitzt.
Dabei sollen folgende Fragen beantwortet werden:

\begin{itemize}[itemsep=0pt]
  \item Wurden die Tabellen-Rechte direkt an dich oder an \texttt{PUBLIC} vergeben?
  \item Welche Rollen besitzt du direkt? Unterscheidet sich zu Aufgabe \ref{sec:uebung_03.aufgabe_01}
  \item Welche Rollen sind anderen Rollen zugeordnet?
  \item Haben Rollen Rechte an \texttt{PESCHM.CONTACTS} oder \texttt{PESCHM.STORAGE}?
\end{itemize}

\subsubsection*{Lösung}
\label{sec:uebung_03.aufgabe_02.loesung}
\inputsql{sql/uebung_03/aufgabe_02.sql}

% ############################### Aufgabe 03 ###############################
\subsection{Aufgabe}
\label{sec:uebung_03.aufgabe_03}
Füge einen neuen Kontakt in die Tabelle \texttt{PESCHM.CONTACTS} ein.

\subsubsection*{Lösung}
\label{sec:uebung_03.aufgabe_03.loesung}
\inputsql{sql/uebung_03/aufgabe_03.sql}

% ############################### Aufgabe 04 ###############################
\subsection{Aufgabe}
\label{sec:uebung_03.aufgabe_04}
Verkürze den Namen der Ressource \texttt{PESCHM.CONTACTS} durch ein \texttt{SYNONYM}.

\subsubsection*{Lösung}
\label{sec:uebung_03.aufgabe_04.loesung}
\inputsql{sql/uebung_03/aufgabe_04.sql}

% ############################### Aufgabe 05 ###############################
\subsection{Aufgabe}
\label{sec:uebung_03.aufgabe_05}
Gebe alle Datensätze mit den Spalten \texttt{FIRSTNAME}, \texttt{LASTNAME}, \texttt{PHONE}, \texttt{MOBILE} und \texttt{MAIL} der Tabelle \texttt{CONTACTS} und der Tabelle \texttt{PESCHM.CONTACTS} unter Angabe deiner verkürzten Schreibweise aus Aufgabe \ref{sec:uebung_03.aufgabe_04} aus. Vermeide bei der Ausgabe doppelte Datensätze.

\subsubsection*{Lösung}
\label{sec:uebung_03.aufgabe_05.loesung}
\inputsql{sql/uebung_03/aufgabe_05.sql}

% ############################### Aufgabe 06 ###############################
\subsection{Aufgabe}
\label{sec:uebung_03.aufgabe_06}
Speichere die Abfrage aus Aufgabe \ref{sec:uebung_03.aufgabe_05} als \texttt{VIEW}.

\subsubsection*{Lösung}
\label{sec:uebung_03.aufgabe_06.loesung}
\inputsql{sql/uebung_03/aufgabe_06.sql}

% ############################### Aufgabe 07 ###############################
\subsection{Aufgabe}
\label{sec:uebung_03.aufgabe_07}
Räume dem Benutzer \texttt{PESCHM} und der Rolle \texttt{FH\_TRIER} das Recht ein, Datensätze aus dem Lager zu abzurufen. Dabei soll der Benutzer \texttt{PESCHM} zusätzlich die Spalte \texttt{CONTACT\_ID} aktualisieren dürfen.

\subsubsection*{Lösung}
\label{sec:uebung_03.aufgabe_07.loesung}
\inputsql{sql/uebung_03/aufgabe_07.sql}

% ############################### Aufgabe 08 ###############################
\subsection{Aufgabe}
\label{sec:uebung_03.aufgabe_08}
Entziehe Benutzer \texttt{PESCHM} und der Rolle \texttt{FH\_TRIER} die unter Aufgabe \ref{sec:uebung_03.aufgabe_07} eingeräumten Rechte.

\subsubsection*{Lösung}
\label{sec:uebung_03.aufgabe_08.loesung}
\inputsql{sql/uebung_03/aufgabe_08.sql}

% ############################### Aufgabe 09 ###############################
\subsection{Aufgabe}
\label{sec:uebung_03.aufgabe_09}
Gebe von allen Mitarbeitern das Geburtsdatum aus und berechne das Alter jedes Mitarbeiters in Jahren, und Monaten. Konkateniere die Spalten soweit wie Möglich zu einer Spalte zusammen wie in Aufgabe \ref{sec:uebung_01.aufgabe_12}.

\begin{info-popup}
  Nutze die Funktion \texttt{months\_between()} und \texttt{mod()} zur Lösung der Aufgabe.

  \begin{sqlcode}
    -- months_between
    SELECT months_between(TO_DATE('10', 'MM'), TO_DATE('04', 'MM'))
    FROM dual;

    -- mod (modulo)
    SELECT mod(9,4)
    FROM dual;
  \end{sqlcode}
\end{info-popup}

\subsubsection*{Lösung}
\label{sec:uebung_03.aufgabe_09.loesung}
\inputsql{sql/uebung_03/aufgabe_09.sql}

% ############################### Aufgabe 10 ###############################
\subsection{Aufgabe}
\label{sec:uebung_03.aufgabe_10}
Gebe das Datum aus, dass heute vor 2 Jahren, 7 Monaten, 24 Tagen, 4 Stunden und 56 Minuten gewesen ist.

\subsubsection*{Lösung}
\label{sec:uebung_03.aufgabe_10.loesung}
\inputsql{sql/uebung_03/aufgabe_10.sql}

% ############################### Aufgabe 11 ###############################
\subsection{Aufgabe}
\label{sec:uebung_03.aufgabe_11}
Gebe für jeden Mitarbeiter die Beschäftigungsdauer in Jahren aus.

\subsubsection*{Lösung}
\label{sec:uebung_03.aufgabe_11.loesung}
\inputsql{sql/uebung_03/aufgabe_11.sql}

% ############################### Aufgabe 12 ###############################
\subsection{Aufgabe}
\label{sec:uebung_03.aufgabe_12}
Erweitere die Tabelle \texttt{EMPLOYEES} um eine Spalte mit dem Gehalt (\texttt{SALARY}) und den Beschäfti-gungsjahren \texttt{HIRE\_YEAR} . Jeder Mitarbeiter erhält ein Einstiegsgehalt von 2000\euro{}. Zusätzlich erhält jeder Mitarbeiter nach Beschäftigungsdauer einen prozentualen Aufschlag gestaffelt nach der folgenden \hyperref[tbl:uebung_03.aufgabe_12]{Tabelle}.

\begin{info-popup}
  Nutze zur Lösung der Verschachtelung ein \texttt{CASE}-Konstrukt. Hier ein gutes Beispiel von \href{https://stackoverflow.com/a/5171127/7652996}{stackoverflow.com}
\end{info-popup}

\begin{table}[H]
  \ttfamily
  \centering
  \begin{tabular}{l|l}
    \textbf{Beschäftigungsdauer} & \textbf{Zuschlag} \\
    \hline\hline
    $0<=x<5$ Jahre & 0\% \\
    $5<=x<7$ Jahre & 5\% \\
    $7<=x<10$ Jahre & 10\% \\
    $10<=x<12$ Jahre & 12,5\% \\
    $12<=x$ Jahre & 15\% \\
  \end{tabular}
  \label{tbl:uebung_03.aufgabe_12}
\end{table}

\subsubsection*{Lösung}
\label{sec:uebung_03.aufgabe_12.loesung}
\inputsql{sql/uebung_03/aufgabe_12.sql}

% ############################### Aufgabe 13 ###############################
\subsection{Aufgabe}
\label{sec:uebung_03.aufgabe_13}
Konkateniere Zeichensätze zu folgendem Satz. \textit{\texttt{FIRSTNAME} ist \texttt{JAHRE} alt}. Gebe diesen Satz für alle Mitarbeiter aus.

\subsubsection*{Lösung}
\label{sec:uebung_03.aufgabe_13.loesung}
\inputsql{sql/uebung_03/aufgabe_13.sql}