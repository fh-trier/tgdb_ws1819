% ##########################################################################
% ############################# Übungsblatt 06 #############################
% ##########################################################################
\section{Übungsblatt}
\label{sec:uebung_06}

% ############################### Aufgabe 01 ###############################
\subsection{Aufgabe}
\label{sec:uebung_06.aufgabe_01}
Schauen Sie sich den nachfolgenden PL/SQL-Code genauer an und geben Sie in kurzen Worten wieder, was der PL/SQL-Code macht.

Ändern Sie anschließend die Ausgabe des PL/SQL-Codes, dass alle Informationen über die abzufragende Person ausgegeben wird.

\inputsql{sql/uebung_06/aufgabe_01_snippet.sql}

\subsubsection*{Lösung}
\label{sec:uebung_06.aufgabe_01.loesung}
\inputsql{sql/uebung_06/aufgabe_01.sql}

% ############################### Aufgabe 02 ###############################
\subsection{Aufgabe}
\label{sec:uebung_06.aufgabe_02}
Schauen Sie sich den nachfolgenden PL/SQL-Code genauer an und geben Sie in kurzen Worten wieder, was der PL/SQL-Code macht.

Ergänzen Sie den PL/SQL-Codeblock so, dass zu erkennen ist, in welchem Bundesland/Stadt ein Lieferant ansässig ist.

\inputsql{sql/uebung_06/aufgabe_02_snippet.sql}

\subsubsection*{Lösung}
\label{sec:uebung_06.aufgabe_02.loesung}
\inputsql{sql/uebung_06/aufgabe_02.sql}

% ############################### Aufgabe 03 ###############################
\subsection{Aufgabe}
\label{sec:uebung_06.aufgabe_03}
Ergänzen Sie Aufgabe \ref{sec:uebung_06.aufgabe_02} so, dass nicht nur das Unternehmen sondern auch die Mitarbeiter mit Vor- und Nachnamen ausgegeben werden.

\subsubsection*{Lösung}
\label{sec:uebung_06.aufgabe_03.loesung}
\inputsql{sql/uebung_06/aufgabe_03.sql}


% ############################### Aufgabe 04 ###############################
\subsection{Aufgabe}
\label{sec:uebung_06.aufgabe_04}
Erstellen Sie einen anonymen PL/SQL-Codeblock, der das Einfügen von neuen Datensätzen vereinfacht.
Fügen Sie über ihren PL-SQL-Codeblock folgende Datensätze ein.

\begin{itemize}[itemsep=0pt]
  \item 5$\times$ Edelstahl, $4000\times2000\times3$, A10, Thyssenkrupp AG
  \item 5$\times$ Edelstahl 14301 Träne, $4750\times500\times10$, B20, Edelmetalle Benelux GBR
  \item 140$\times$ Messing, $2000\times2000\times5$, C70, Kruppstahl AG
\end{itemize}

Fehlende Informationen können frei ergänzt werden.

\subsubsection*{Lösung}
\label{sec:uebung_06.aufgabe_04.loesung}
\inputsql{sql/uebung_06/aufgabe_04.sql}